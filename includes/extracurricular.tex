\cvsection{Publicly accessible projects}

\begin{cventries}

  \cventry
  {Objective-C, C}
  {\href{https://github.com/pallotron/yubiswitch}{Yubiswitch}}
  {\href{https://github.com/pallotron/yubiswitch}{github}}
  {}
  {
    A macOS status bar app to toggle Yubikey's OTP functionality.
    Mentioned in YubiCo's \href{https://www.yubico.com/blog/yubiswitch/}{blog post}.
    Initially a personal project, it gained popularity among users and tech employees at Meta, Apple, Google, and other big tech companies.
  }

  \cventry
  {golang}
  {\href{https://github.com/facebookincubator/dhcplb}{dhcplb}}
  {\href{https://github.com/facebookincubator/dhcplb}{github}}
  {}
  {
    A DHCP load balancer and server running at Meta's datacenters.
    For the full story of how the project was born and went from hack-a-thon project, to
    summer internship to production read \href{https://engineering.fb.com/2016/09/13/data-infrastructure/dhcplb-an-open-source-load-balancer/}{this post in Meta's engineering blog}.
  }

  \cventry
  {python}
  {\href{https://github.com/facebookarchive/fbtftp}{fbtftp}}
  {\href{https://github.com/facebookarchive/fbtftp}{github}}
  {}
  {
    fbtftp is Facebook's implementation of a dynamic TFTP server framework. It lets you create custom TFTP servers and wrap your
    own logic into it in a very simple manner. Facebook used it in production, and was deployed at global scale
    across all of data centers before being replaced by a golang implementation first, and a rust implementation later.
  }

\end{cventries}

\cvsection{Extracurricular Activity}

\begin{cventries}
  \cventry
  {Long distance cycling (aka randonneuring/audax)}
  {}
  {}
  {}
  {
    I have completed several 200km, 300km, 400km, and 600km rides.
    I have particpated to the ~1200km Paris-Brest-Paris twice: once in 2019 (had to scratch after 600km)
    and in 2023 (finished with a few hours to spare).
  }

  \cventry
  {Volunteered to local LinuxUser Group}
  {}
  {}
  {}
  {
    In the early 2000s, I volunteered at the Linux User Group in Catania, Italy.
    I helped organize events to promote Open Source/Free Software, gave talks about Linux and open source software in
    schools and at the university, and helped setting up labs with Linux computers in schools using cheap hardware and free software.
  }
\end{cventries}