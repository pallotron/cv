\cvsection{Experience}

\begin{cventries}

  %---------------------------------------------------------
  \cventry
  {Staff Production Engineer (IC6)} % Job title
  {Meta(Facebook)} % Organization
  {Dublin, Republic of Ireland} % Location
  {March 2011 - Present} % Date(s)
  {
    Progressed from IC3 to IC6, advancing from being an IC3 on a Tier 1 with infrastructure-wide support responsibilities to
    driving critical datacenter automation initiatives in the bare metal provisioning space.
    Recently, I have led efforts to transform the way
    we manage the Operating System for Meta's custom network switch infrastructure
    \href{https://engineering.fb.com/2015/03/10/data-center-engineering/facebook-open-switching-system-fboss-and-wedge-in-the-open/}{(FBOSS)},
    focusing on scalability, reliability, ownership, and time-to-provision improvements.
  }
  \vspace{5mm}

  \cventryprevrole
  {Production Engineer (IC6) - Team(s): Network OS}
  {2022 \newline Present}
  {
    In 2022, I started collaborating with people in the US on efforts to redesign the Operating System management approach at Meta, transitioning from a traditional model of provisioning a vanilla OS followed by configuration with Chef at first boot (more time-intensive and less efficient), to an immutable image-based model which
    significantly reduces provisioning time and increases reliability.
    \vspace{2mm}
    \begin{cvitems}
      \item Recognizing the challenges of organizational adoption, \textbf{I successfully advocated for the creation of a dedicated team} to implement this new approach for Meta's custom network switch infrastructure (FBOSS). This required \textbf{extensive collaboration and alignment with stakeholders across various levels}, including managers, directors, VPs, and ICs, to emphasize the critical need for this transformation to enhance reliability, scalability, and security.
      \item The initial prototype demonstrated a dramatic improvement, \textbf{reducing provisioning time from 2 hours to 15 minutes.}
      \item This success \textbf{catalyzed enhancements in key operational processes}, including rack ingestion, rack moves, and repair workflows, ultimately driving greater efficiency and reliability across the network infrastructure.
    \end{cvitems}
  }

  \cventryprevrole
  {Production Engineer (IC4->IC6) - Team(s): ClusterOps -> ClusterInfra -> Host Provisioning Engineering -> DC Bootstrap}
  {April 2014 \newline 2022}
  {
    Following the successful automation of the SRO team's core responsibilities, the team was strategically dissolved and
    members were presented with opportunities to transition into new specialized focus areas.
    While several colleagues moved to product teams, I chose to remain in infrastructure by joining the ClusterOps team.
    This team managed essential infrastructure services including DNS, LDAP, NTP, DHCP, TFTP, and bare metal provisioning systems.
    \newline
    During this period, I achieved several significant technical milestones:
    \vspace{2mm}
    \begin{cvitems}
      \item {\textbf{DHCP Infrastructure Modernization}: Designed and led the complete redesign of Meta's DHCP infrastructure,
                  implementing ISC KEA with custom C++ hooks for internal system integration. This initiative
                  significantly improved reliability and scalability. Presented the architecture at \href{https://www.usenix.org/conference/srecon15europe/program/presentation/failla}{SRECON EMEA 2015}
                  and was featured in \href{https://www.isc.org/blogs/how-facebook-is-using-kea-in-the-datacenter/}{the ISC's blog.}}
      \item {\textbf{IPv6-only Initiative}: Participated to a critical infrastructure project to transition
                  data centers to IPv6-only addressing, addressing RFC 1918 address space constraints. In that context I led the
                  adaptation of the PXEboot stack and OS installation systems to support IPv6-only environments, contributing
                  to Meta's pioneering IPv6 adoption \href{https://www.internetsociety.org/resources/deploy360/2014/case-study-facebook-moving-to-an-ipv6-only-internal-network/}{and praised by the Internet Society}.
                  This enabled the migration and scaling up to hosting millions of servers.}
      \item {\textbf{Designed and implemented \texttt{dhcplb}}, a high-performance
                  DHCP load balancer in Go. The project evolved from a hackathon prototype to a production-grade
                  solution, featuring IPv6 support and sophisticated load balancing algorithms. The initiative spawned
                  an industry-adopted open-source DHCP library and later expanded to include server-side functionality,
                  becoming a cornerstone of Meta's network infrastructure.}
      \item {\textbf{Modernized the PXEboot infrastructure, leading complete rewrite from Python
                    to Go} to enhance performance and reliability and support ever growing demand/capacity.}
      \item {\textbf{Led the development of a custom OS installation solution}, replacing traditional RedHat
                  anaconda/kickstart systems with an optimized, more highly integrated with Facebook's infra, purpose-built installer utilizing custom bootloader, ramdisk and an initrd based on \href{https://github.com/u-root/u-root}{u-root}.}
    \end{cvitems}
  }

  \cventryprevrole
  {Site Reliability Engineer (IC3->IC4) - Team(s): SRO}
  {
    March 2011 \newline April 2014
  }
  {
    I began my tenure at Facebook in 2011 as a Site Reliability Engineer (SRE) within the Site Reliability Operations (SRO) team. The SRO team operated on a follow-the-sun model, with a counterpart team in Palo Alto, ensuring round-the-clock reliability of Facebook's infrastructure. This role provided me with extensive exposure to internal technologies and systems, and honed my skills in managing large-scale systems and handling incidents and disasters. Being based in Dublin allowed me to focus on automation and tooling development during quieter hours.
    \vspace{2mm}
    \begin{cvitems}
      \item {Contributed in developing internal tooling like \href{https://engineering.fb.com/2011/09/15/data-center-engineering/making-facebook-self-healing/}{FBAR}, Facebook Auto Remediation, HostChangeLog, etc.}
      \item {Drove emergency hotfixing and pushing of the www application durnig Dublin hours.}
      \item {Contributed developing cluster maintenances tools and perform critical maintenances like kernel updates, OS updates, host reprovisioning, and others.}
      \item {Diagnosed and resolved complex hardware and software issues.}
      \item {Contributed to the migration of the fleet from \texttt{cfengine2} to \texttt{chef}.}
      \item {Architected and implemented Facebook's initial \texttt{netconsole} monitoring system for kernel-level diagnostics.
                  The solution leveraged the Linux netconsole module to capture kernel printk messages via UDP/syslog, enabling critical
                  debugging capabilities when disk logging was unavailable. Designed and deployed a scalable architecture using
                  cluster-based rsyslog daemons integrated with \texttt{scribe} infrastructure. This foundational work later evolved
                  into \href{https://github.com/facebook/netconsd}{netconsd}, now maintained by the Kernel Platform Engineering team.}
    \end{cvitems}
  }

  \cventry
  {Senior Deployment Engineer} % Job title
  {NewBay Software} % Organization
  {Dublin, Republic of Ireland} % Location
  {February 2008 - March 2011} % Date(s)}
  {
    Newbay Software provided digital content services, like photo storage, blogging and social aggregator platforms,
    primarily for telecommunication clients like T-Mobile and Telstra.
    The company was acquired by RIM and subsequently sold to Synchronoss Technologies.
    \vspace{2mm}
    \begin{cvitems} % Description(s) of tasks/responsibilities
      \item Managed software deployment and customer integration projects for major telecommunication clients, including T-Mobile (Germany/US) and Telstra (Australia).
      \item Provided advanced troubleshooting and issue resolution for MMS, SMS, standalone, and web-based applications.
      \item Developed and maintained software deployment tooling.
      \item Executed front-end deployments, ensuring seamless user experience and functionality.
      \item Configured and deployed VPNs with various telecommunication companies to establish secure SMS/MMS gateways.
      \item Proficient in a diverse technology stack, including Cisco IPSEC VPNs and networks, Oracle/MySQL databases, Java Messaging Systems, F5 BigIP Load Balancers, Red Hat and Solaris operating systems, and Tomcat/JBoss application servers.
    \end{cvitems}
  }

  \cventry
  {Unix System and Network Administrator, Programmer and IT Consultant} % Job title
  {MOVIA SpA} % Organization
  {Catania, Italy} % Location
  {September 2005 - January 2008} % Date(s)
  {
    \begin{cvitems} % Description(s) of tasks/responsibilities
      \item Administered and maintained the company's corporate network infrastructure across three Italian offices (Rome, Milan, Catania).
      \item Managed and secured the corporate intranet utilizing OpenVPN and Cisco routers.
      \item Oversaw the company's web presence, mail servers, mailing lists, version control systems (CVS), and internal ticketing systems.
      \item Provided expert consultation to Nokia Italy, Vodafone Italy, and LogicaCMG on an as-needed basis.
    \end{cvitems}
  }

  \cventry
  {Unix Sysadmin and Network Administrator}
  {ComputerLine SRL}
  {Catania, Italy}
  {December 2004 - September 2005}
  {
    Small web agency providing web hosting, web design, and IT consulting services to local businesses.
    \begin{cvitems}
      \item Ordinary maintenance of internet services like DNS server, Web servers, FTP servers, Mail servers etc.
      \item DNS domains registration and maintenance.
      \item Troubleshooting and, technical support for customers and internal staff.
    \end{cvitems}
  }

  \cventry
  {L.A.M.P. Web developer and system administrator}
  {UZED@ SRL}
  {Catania, Italy}
  {December 2004 - September 2005}
  {
    Small web agency providing web hosting, web design, and IT consulting services to local businesses.
    \begin{cvitems}
      \item Ordinary maintenance of internet services like DNS server, Web servers, FTP servers, Mail servers etc.
      \item DNS domains registration and maintenance.
      \item Troubleshooting and, technical support for customers and internal staff.
      \item Developed web applications for customers using PHP and MySQL.
    \end{cvitems}
  }

  \cventry
  {Web developer}
  {University of Catania, Department of Engineering}
  {Catania, Italy}
  {January 2001 - December 2002}
  {
    I was responsible for building and maintaining the students e-learning portal.
  }

  \cventry
  {Visual Basic developer}
  {Infozoo Project}
  {Catania, Italy}
  {January 2001 - March 2002}
  {
    Small software house developing software for the retail sector.
    I have worked as a Visual Basic coder, my mainly responsibility was writing drivers to enable
    communication between electronic scales (on RS232).
  }

\end{cventries}
