%-------------------------------------------------------------------------------
%	SECTION TITLE
%-------------------------------------------------------------------------------
\cvsection{Work Experience}


%-------------------------------------------------------------------------------
%	CONTENT
%-------------------------------------------------------------------------------
\begin{cventries}

  %---------------------------------------------------------
  \cventry
  {Staff Production Engineer (IC6)} % Job title
  {Meta(Facebook)} % Organization
  {Dublin, Republic of Ireland} % Location
  {March 2011 - Present} % Date(s)
  {
    I naturally gravitated towards core infrastructure, and worked on internal core systems like DNS, LDAP, NTP, monitoring tooling, and
    tooling to help with data center automation tooling (bare metal provisioning space).
    I have driven various initiatives to improve the performance and reliability of our core infrastructure services bringing
    down bare metal provisioning times.
    Recently I have moved to the Network OS team, working on redesigning the network operating system for our FBOSS switches.
  }

  \cventryprevrole
  {Production Engineer (ClusterOps -> ClusterInfra -> Host Provisioning Engineering -> DC Bootstrap)}
  {April 2014 \newline 2020}
  {
    After the SRO team automated itself out of existence thanks to automation, I moved to the ClusterOps team.
    The ClusterOps team was responsible for core infrastructure services like DNS, LDAP, NTP, DHCP, TFTP and internal services
    to orchestrate bare metal provisioning.
    \begin{cvitems}
      \item {DHCP infrastructure hoverhaul: In 2013/2014 I rewrote the DHCP infrastructure from scratch, replacing the old
                  ISC DHCPd with a new system that was more reliable, scalable and easier to manage. The system was based on ISC KEA
                  and implemented C++ hook to integrate with Facebook's internal systems.I made a presentation about it
                  at SRECON EMEA 2015, and an interview on the ISC blog.}
      \item \texttt{dhcplb}:  I wrote most of the code that runs Facebook’s DHCP infrastructure, including
      an open source DHCP library at https://github.com/insomniacslk/dhcp that is used by several big tech
      companies. After the rewrite, this became one of Meta’s most reliable services, with near-zero
      operations. The cleaner structure made it easy to modify and test by other teams, simplifying the
      creation of new, specialized use cases.
      \item {Core infra services:}
    \end{cvitems}
  }

  \cventryprevrole
  {Site Reliability Engineer (SRO team)}
  {
    March 2011 \newline April 2014
  }
  {
    I joined Facebook in 2011 as a Site Reliability Engineer (SRE) in the Site Reliability Operations (SRO) team.
    The SRO team was a follow the sun team, with a sister team in Palo Alto, responsible for being the first
    responder for the reliability of the entire Facebook infrastructure.
    This gave me the opportunity to learn a lot about internal technologies and systems.
    I learned how to manage systems at scale, how to deal with incidents/disasters.
    Being based in Dublin gave me the opportunity to work during quiet mostly incident free hours, which has allowed
    me to focus on automation and tooling.
    \vspace{2mm}
    \begin{cvitems}
      \item {Contributed to the development of internal tooling like \href{https://engineering.fb.com/2011/09/15/data-center-engineering/making-facebook-self-healing/}{FBAR}, Facebook Auto Remediation, HostChangeLog, etc}
      \item {Performed emergency hotfixing and pushing of the www application}
      \item {Performed cluster maintenances: kernel updates, OS updates, host reprovisioning, and others}
      \item {Troubleshooted broken hardware and software issues}
      \item {Contributed migrating fleet from \texttt{cfengine2} to \texttt{chef}}
      \item {Architected and implemented Facebook's initial \texttt{netconsole} monitoring system for kernel-level diagnostics.
                  The solution leveraged Linux netconsole module to capture kernel printk messages via UDP/syslog, enabling critical
                  debugging capabilities when disk logging was unavailable. Designed and deployed a scalable architecture using
                  cluster-based rsyslog daemons integrated with \texttt{scribe} infrastructure. This foundational work later evolved
                  into \href{https://github.com/facebook/netconsd}{netconsd}, now maintained by the Kernel Platform Engineering team.}
    \end{cvitems}
  }

  \cventry
  {Senior Deployment Engineer} % Job title
  {NewBay Software} % Organization
  {Dublin, Republic of Ireland} % Location
  {February 2008 - March 2011} % Date(s)}
  {
    Newbay Software provided digital content services, like photo storage, blogging and social aggregator platforms,
    primarily for telecommunication clients like T-Mobile and Telstra.
    The company was acquired by RIM and subsequently sold to Synchronoss Technologies.
    \vspace{2mm}
    \begin{cvitems} % Description(s) of tasks/responsibilities
      \item Managed software deployment and customer integration projects for major telecommunication clients, including T-Mobile (Germany/US) and Telstra (Australia).
      \item Provided advanced troubleshooting and issue resolution for MMS, SMS, standalone, and web-based applications.
      \item Developed and maintained software deployment tooling.
      \item Executed front-end deployments, ensuring seamless user experience and functionality.
      \item Configured and deployed VPNs with various telecommunication companies to establish secure SMS/MMS gateways.
      \item Proficient in a diverse technology stack, including Cisco IPSEC VPNs and networks, Oracle/MySQL databases, Java Messaging Systems, F5 BigIP Load Balancers, Red Hat and Solaris operating systems, and Tomcat/JBoss application servers.
    \end{cvitems}
  }

  \cventry
  {Unix System and Network Administrator, Programmer and IT Consultant} % Job title
  {MOVIA SpA} % Organization
  {Catania, Italy} % Location
  {September 2005 - January 2008} % Date(s)
  {
    \begin{cvitems} % Description(s) of tasks/responsibilities
      \item Administered and maintained the company's corporate network infrastructure across three Italian offices (Rome, Milan, Catania).
      \item Managed and secured the corporate intranet utilizing OpenVPN and Cisco routers.
      \item Oversaw the company's web presence, mail servers, mailing lists, version control systems (CVS), and internal ticketing systems.
      \item Provided expert consultation to Nokia Italy, Vodafone Italy, and LogicaCMG on an as-needed basis.
    \end{cvitems}
  }

  \cventry
  {Unix Sysadmin and Network Administrator}
  {ComputerLine SRL}
  {Catania, Italy}
  {December 2004 - September 2005}
  {
    Small web agency providing web hosting, web design, and IT consulting services to local businesses.
    \begin{cvitems}
      \item Ordinary maintenance of internet services like DNS server, Web servers, FTP servers, Mail servers etc.
      \item DNS domains registration and maintenance.
      \item Troubleshooting and, technical support for customers and internal staff.
    \end{cvitems}
  }

  \cventry
  {Web developer and system administrator}
  {UZED@ SRL}
  {Catania, Italy}
  {December 2004 - September 2005}
  {
    Small web agency providing web hosting, web design, and IT consulting services to local businesses.
    \begin{cvitems}
      \item Ordinary maintenance of internet services like DNS server, Web servers, FTP servers, Mail servers etc.
      \item DNS domains registration and maintenance.
      \item Troubleshooting and, technical support for customers and internal staff.
    \end{cvitems}
  }

  \cventry
  {Web developer}
  {University of Catania, Department of Engineering}
  {Catania, Italy}
  {January 2001 - December 2002}
  {
    I was responsible for building and maintaining the students e-learning portal.
  }

  \cventry
  {Visual Basic developer}
  {Infozoo Project}
  {Catania, Italy}
  {January 2001 - March 2002}
  {
    Small software house developing software for the retail sector.
    I have worked as a Visual Basic coder, my mainly responsibility was writing drivers to enable
    communication between electronic scales (on RS232).
  }

\end{cventries}
